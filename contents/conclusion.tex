% # -*- coding: utf-8 -*-
\section{Conclusion}  \label{section:conclustion}
%
% 本稿では,特定のカーネルデータへの読書き制限を制御し,権限情報を保護を可能とす
% るセキュリティ機構を提案した.提案したセキュリティ機構においては,CPUの提供す
% るメモリ保護機構として, PKSを用いた PTE への読書き制限を動作中のカーネルにお
% いて制御可能とした.
% 提案したセキュリティ機構では,CPUの提供するメモリ保護機構であるMPK PKSを用いるこ
% とで動作中のカーネルにおいて指定したカーネルデータの書込み制限を制御可能としてい
% る.

An adversary can achieve privilege escalation and the defeat of security
mechanisms by corrupting the kernel memory.
% through kernel memory corruption.
%
KCoFI, KASLR, and AKO are kernel attack countermeasures that mitigate and
prevent the threat of kernel attacks.
%
However, vulnerable kernel codes can still modify the kernel data at the kernel layer.

%
In this paper, we proposed a novel security design of a KDPM that manages write
restrictions on specific kernel data.
%and enables protection of authorization information. 
The KDPM enables the kernel to control write privileges on PTEs using the MPK PKS
in the running kernel by the CPU.
% It can control the write limit of specified kernel data in the running kernel by the CPU.
%
% 提案したセキュリティ機構において,カーネルデータへの書込み制限を動的制御するセ
% キュリティ機構の設計を整理し,権限情報を保護するための実現方式1,ならびにカー
% ネルのセキュリティ機能を保護する実現方式2について機能考察と既存研究との比較考
% 察を進め,性能評価を行った.
% We designed the proposed security mechanism and 
We compared the KDPM with existing approaches.
% that dynamically controls write
% restrictions on kernel data, 
% existing research and evaluated the performance of Scheme 1, which protects
% authorization information, and Scheme 2, which protects the kernel security
% functions.
%
% 実現したセキュリティ機構においては,システムコール単位で権限情報の読書き制限を
% 制御し,攻撃を行うユーザプロセスに対して,権限情報への書込み制限を実現した.
% また,カーネルコード単位で強制アクセス制御のカーネルデータに対する読書き制限を
% 制御し,攻撃を行うユーザプロセスによる強制アクセス制御の無効化の制限を実現した.
%
From the two implementations of the KDPM, Implementation 1 protects the
privileged information of the user process to prevent privilege escalation,
whereas Implementation 2 protects the kernel data of the MAC to prevent the
defeat of security mechanisms.
% functions.

% write restriction of authorization
% information is controlled on a per-system-call basis, and write restriction to
% authorization information is realized for the attacking user process. In
% addition, the kernel code unit controls the read restrictions on the kernel data
% of the forced access control, thereby restricting the invalidation of the forced
% access control by the user process that is attacking the kernel.
%
% 本稿では,メモリ破壊攻撃への対策として,我々の提案しているカーネルデータへの書
% 込み制限を動的制御するセキュリティ機構の設計を整理し,権限情報を保護するための
% 実現方式1,ならびにカーネルのセキュリティ機能を保護する実現方式2について機能考
% 察と既存研究との比較考察を進め,性能評価を行った.
%
% Vulnerable kernel codes can open a system to attacks via kernel memory
% corruption, which can be used for privilege escalation or the defeat
% of security features.
% %
% Although kRazor and KASR accomplish attack surface reduction
% using kernel code tracing methods to reduce kernel codes visible to
% benign user processes,
% %
% these approaches require the preparation of sets of kernel code lists
% for the stable operation of benign applications and the
% analysis of full kernel behavior.

% vkTracer, however, presents a novel approach for the tracing and
% identification of vulnerable codes.
% %
% It involves the collection of an invoked vulnerable kernel code
% with a virtual address and function name from the user process of a
% PoC code.
% %
% Subsequently, vkTracer provides a profile of vulnerable kernel
% codes.
% %
% Therefore, it only requires a short program (PoC) code and dynamically analyzes
% parts of the running kernel and kernel debug information. vkTracer responds to
% the latest kernel vulnerabilities to restrict the kernel attack surface.
%
% 評価においては,提案手法を実現したLinuxに対し,特権奪取攻撃に利用可能
% なカーネル脆弱性を評価用に導入し,ユーザプロセスからの特権奪取攻撃によ
% る権限情報の書込み制限を実現可能であることを示した.
%
% 性能評価においては,実現方式1を実装したLinuxにて,システムコール呼出しに対して
% は, 2.96\% から 9.01\% のオーバヘッドに留まること,
% %
% ならびに MPK PKS操作にかかる負荷は 22.1 ns から 1347.9 ns のオーバヘッドとなるこ
% とを明らかにした.
%
In the evaluation, we introduced a kernel vulnerability that can be exploited for
privilege escalation attacks and demonstrated the restriction capability for the
writing of privileged information of the user processes.
%
%  into Linux, which implements the proposed method, for
% evaluation, and showed that it is feasible to restrict the writing of privilege
% information by privilege-grabbing attacks from user processes.
% %
The performance evaluation showed that the overhead for invoking system call 
on Linux with Implementation 1 ranged from 2.96\% to 9.01\%, and the PKS
operations overhead on Linux with Implementation 2 ranged from 22.1 ns to 1347.9
ns.

%
In future studies, to prevent vulnerable kernel code execution and illegal
modification of kernel data due to the principle of security risk and
performance overhead, researchers can provide the design of lightweight security
mechanism that combines the verification of kernel code execution sequence and
the write protection of kernel data at the adequate timing to mitigate kernel
attacks.
% ensure sufficient attack prevention on the running kernel.
%
% The evaluation results show that the proposed method supports several types of kernel
% vulnerabilities in identifying vulnerable kernel codes.
% %
% The performance cost of PoC code tracing was from 5.2683 s to 5.2728 s for
% each kernel vulnerability.
% %
% Moreover, the performance overhead of the maximum system call
% invocations was 3.7197 $\mu$s.
% %
% The overhead to a web client program averages from 0.37 \% to 0.56
% \% for HTTP download access of 100,000 HTTP sessions.
%%   %
%vkTracer represents 2.459 \% to 2.193 \% as the dynamic kernel tracing
%overhead. 
%% The vulnerable kernel codes induce kernel memory corruption
%%   that causes the privilege escalation or the defeating of security
%%   features.
%%   %
%%   However, although kRazor and KASR are attack surface reduction with
%%   tracing kernel code methods of benign user process to reduce the
%%   visible kernel codes,
%%   %leads kernel memory corruption of 
%%   %
%%   these approaches prepare set of kernel code list for the stable
%%   working of benign applications, which require the analyzing of full
%%   kernel behavior.
  %kernel.
  %and statically assing of kernel image.
%}
%the security mechanisms of the novel


%and protect privilege data from memory corruption.
%}
%
%% \blueuline{
%%   Additionally,
%%   The performance overhead of the maximum system call invocations
%%   was 0.703 $\mu$s.
%%   %
%%   The overhead to a web client program averages
%%   from 1.188 \% to 4.093 \% for HTTP download access of
%%   100,000 HTTP sessions.
%%   %
%%   The implementations of KPRM represents 2.459 \% and 2.193 \% as the
%%   kernel compiling time overhead.}
%% \blueuline{[N-09]
%%   The effect of kernel memory corruption that leads to the 
%%   the privilege escalation or the defeats security features.
%%   %
%%   The kernel protection mechanisms focus on mitigating and preventing
%%   the actual thret for the running kernel.
%% %
%%   %The kernel adopts several countermeasures, including stack monitoring,
%%   The stack monitoring,  CFI, KASLR, KPTI, Proclocal, and SCI
%%   are kernel attack prevention countermeasures.
%%   %
%%   However, vulnerable kernel code, and kernel code or kernel data
%%   still share the kernel address space.
%
%\reduline{
  %then automatically identify virtual address and function name of vulnerable kernel code.
  %and kernel data for the adversary's user process.
  %
  %% %the restricted kernel pages to manage vulnerable kernel code and attack target kernel code.
  %% the restricted kernel pages to manage vulnerable kernel code and attack target kernel data.
  %% %
  %% The KPRM dynamically unmap restricted kernel pages from the kernel
  %% page table for the adversary's user process with the profile of
  %% vulnerable kernel code list.
%\reduline{
%% In this paper, the novel security mechanisms of vkTracer
%% presents the collecting and identifying of invoked vulnerable kernel code
%% with virtual address and function name from the user process of PoC code.
%% %}  
%% %
%% %\reduline{
%% %}
%%   %\reduline{
%% Subsequently, vkTracer provides the profile of vulnerable kernel codes.
%%   Therefore, vkTracer only requires the small program code as PoC code and
%%   dynamically analyzes parts of running kernel and kernel debug information.
%%   %with target kernel vulnerability.
%%   %the adversary's user process cannot invoke vulnerable kernel code
%%   %and access restricted kernel data on the running kernel.
%%   It quickly corresponds for the latest kernel vulnerability to
%%   restrict the kernel attack surface.
%% %It completely mitigate the kernel memory corruption.
%% %}
%% %
