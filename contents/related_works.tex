% # -*- coding: utf-8 -*-
\section{Related Work}  \label{section:related_works}
% \begin{figure*}[t]
% %  \begin{center}
%   \centering
%   \hspace*{-4.5ex}
%     %\includegraphics[scale=.275]{./imgs/002_screen_shot_2019-08-04_15.19.54.eps}
%     \includegraphics[bb=0 0 1346 735, scale=.300]{./imgs/A01X_08_screen_shot_2021-08-25_11.30.14.png}
% %  \end{center}
% %  \vspace{-1.0ex}
%   \caption{
%     %
%     Comparison of the tracing mechanism design.
%     %
%   }
% %  \vspace{-2.0ex}  
%   \label{fig:kernel_tracing_mechanisms}
% %  \vspace{-2.0ex}
% \end{figure*}


% {\bf アプリケーションにおけるMPKを利用したデータ保護}\\
{\bf User Process Data Protection using the MPK:}
%\subsubsection{User Process Data Protection with the MPK:}
% libmpk はユーザプロセスでの保護対象データの操作を柔軟化の
% ため,PSU 管理の抽象化を提案している \cite{libmpk}.
% アプリケーションにおける MPK を利用したデータ保護として,
% libmpk ユーザプロセスにおいて,PSUを利用した保護対象データの操作を
% 柔軟にするための抽象化ライブラリを提案している \cite{libmpk}.
For data protection using the MPK in applications, libmpk provides a flexible
library that supports user processes. This can manipulate the protected data
using the PSU \cite{libmpk}.
% A library is proposed to make it flexible 
% %
% MPK を利用したデータ保護として,ERIM は PSU を用いてアプリケーション向
% けに保護対象データを異なるユーザプロセスへ分離する手法を提案している
% \cite{erim}.
% ERIM はPSUを利用してアプリケーションにおいて,複数のユーザプロセスに保護対象データを
% 分離して管理する手法を提案している\cite{erim}.
% %
ERIM is proposed as a separation method for the protected user process data into
different user processes using PSU \cite{erim}.
%
Cerberus is proposed as a sandbox framework for user application using the PSU
\cite{Voulimeneas22eurosys}.

% {\bf カーネルにおけるMPKを利用したデータ保護}\\
{\bf Kernel Data Protection using the MPK:}
%\subsubsection{Kernel Data Protection with the MPK:}
% また,xMP はカーネルの仮想記憶空間を構成する複数のページをドメイン単位
% として Protection key 毎に分割し,仮想マシンモニタ(VMM)から
% 管理する機構を提案している \cite{xmp}.
%
% カーネルにおける MPK を利用したカーネルコードおよびカーネルデータの保護として,
% xMP は複数のドメインを用意し,カーネルの仮想記憶空間を構成する複数のページを各ド
% メインに割当て,Pkey により,仮想マシンモニタ(VMM)から管理する機構を提案してい
% る\cite{xmp}.
To protect the kernel code and kernel data using the MPK in the kernel, xMP
proposes a security mechanism that provides multiple domains. These contain
pages of kernel memory space that are allocated using the PKU. The virtual
machine monitor (VMM) manages domains via Pkeys \cite{xmp}.
% comprising the
% kernel virtual storage space are allocated to each domain, and 
% %
% libhermitMPK はカーネルコードとデータを複数の Pkey に分割管理し,Pkey単位で
% 管理することで不正な読書きから保護する機構を提案している\cite{libhermitmpk}.
Additionally, libhermitMPK proposes a security mechanism to protect against
unauthorized reading and writing by dividing and managing the kernel code and
data into multiple Pkeys\cite{libhermitmpk}.
%
UnderBridge applies MPK for a microkernel between user space and kernel space at
runtime isolation of IPC mechanism \cite{gu20atc}.
%
EPK adopts virtualization features to increase the number of Pkeys in MPK for a
guestOS of microkernel \cite{gu22atc}.

% and managing them on a per-Pkey basis 
%
% libhermitMPK は複数の Protection key にカーネルコードとデータを分割し保護する機
% 構を提案している\cite{libhermitmpk}.

% {\bf 仮想記憶空間の分離によるカーネルデータの保護}
% Kernel page table isolation (KPTI)はユーザプロセスによるサイドチャネ
% %ル攻撃からのカーネルモードのカーネルデータの参照を防止するため,ユーザ
% ル攻撃からカーネルデータの参照を防止するため,ユーザモード用の仮想記憶
% 空間に一部のカーネルコードおよびカーネルデータ,大部分をカーネルモード
% 用の仮想記憶空間に配置した \cite{gruss17ssos}.
% %
% また,Proclocal はユーザプロセス毎にカーネルの仮想記憶空間に占有の記憶
% 領域を割当て \cite{proclocal} ,Address space isolation (ASI)
% \cite{asi} は仮想マシン実行専用のカーネルの仮想記憶空間を備え,仮想化
% に関するカーネルデータの保護を実現している.


% {\bf 不正コード実行防止}\\
{\bf Prevention of Malicious Code Execution:}
%\subsubsection{Prevention of Malicious Code Execution:}
% カーネルにおける不正なカーネルコードの実行防止として,コード呼出し順を検査する
% CFI \cite{abadi05ccs}の適用が進められている\cite{cfi-lwn}.
% また,カーネルへのCFI適用のため,KCoFI では,独自アーキテクチャとして非同期処理
% 発生時におけるコード呼出順の整合性保存機構を提案している\cite{criswell14sp}.
% 脆弱性を利用した攻撃による不正コード実行に対し,コード呼出し順を検査
% する control flow integrity(CFI)が提案されている \cite{abadi05ccs}.
% %カーネルでは,割込み制御等の非同期処理への対応が必
% %要なため,予め決められたコード呼出し順での CFI は困難である.
% KCoFI では,カーネルへのCFI適用のため,独自アーキテクチャとして非同期
% 処理発生時におけるコード呼出順の整合性保存機構を提案している
% \cite{criswell14sp}.
To prevent illegal kernel code execution in the kernel or hypervisor, the
control flow integrity (CFI) \cite{abadi05ccs, wang10sp, ge16eurosp}, which
verifies the order of program function calls, is applied \cite{cfi-lwn}. 
% it is being 
%
To apply the CFI to the kernel, KCoFI is proposed as a security mechanism for
preserving the integrity of the order of invoking  kernel codes as the original
architecture \cite{criswell14sp}.
%
Additionally, pointer authentication based CFI achieves the protection of kernel
execution context with low overhead using ARM hardware feature
\cite{yoo22usenix}.

% when asynchronous processing occurs 


% {\bf Fault Tolerance}  %The stability of kernel behavior requires the
% A stable kernel behavior requires the main kernel feature to be protected from
% identified malicious or buggy device drivers.
% %
% The user space driver includes a fault-tolerance mechanism that separates the
% device driver execution from the main kernel processing
% \cite{herder09dsn,butt09acsac}.
% %
% iKernel adopts a virtual machine monitor that separates buggy devices on virtual
% machines \cite{tan07dasc}.
% %
% SIDE provides a dedicated page table for each driver with interaction
% mechanisms between the kernel and drivers \cite{sun13dsn}.

% %\subsection{Memory Protection}
% %% \end{figure}
% {\bf Attack Surface Reduction}
% %
% Reducing the kernel attack surface restricts the visible virtual
% memory region for user processes.
% %git 
% %% The design was proposed by Dautenhahn {\it et al}.

% %% %
% PerspicuOS demonstrates privilege minimization by manually isolating the
% kernel mechanism for hardware management \cite{dautenhahn15asplos}.
% %
% kRazor manages the list of kernel codes that are visible for user processes
% \cite{kurmus14dimva}.
% %
% KASR handles and controls the kernel page table to reduce the set of
% kernel codes and data for each user program \cite{zhang18arxiv}.

% Multik customizes the kernel image to create a profile that
% contains the necessary kernel code for each traced application \cite{kuo19arxiv}.
% %
% KHide restricts the granularity of software diversity techniques for
% kernel code and kernel data with hardware virtualization \cite{gionta16cns}.

% {\bf Kernel Vulnerability Countermeasures}
% Kernel vulnerability countermeasures prevent the invocation of
% vulnerable kernel codes or restrict illegal kernel code behaviors.
% %
% KCoFI corresponds with control-flow integrity for kernel processing, which
% requires asynchronous behavior to handle interruptions and context switches
% \cite{criswell14sp}.
% %
% eXclusive Page Frame Ownership manages the attribution separation of
% the page between the user and kernel modes to protect the direct
% mapping region attacks \cite{kemerlis14usenix}.
% %
% kR\^{}X controls the exclusive mechanism across the access and
% execution privileges of the kernel code and kernel data \cite{pomonis17eurosys}.


%PerspicuOS that demonstrates privilege minimization to manually
%isolate the kernel mechanism for hardware management
%to assign for kernel isolation mechanism 
%KASR that handles and controls kernel page table to reduce a set of
%kernel code and kernel data for each user program execution
%\cite{zhang18arxiv}.
%
%\cite{kurmus14dimva,zhang18arxiv}.
%kRazor and KASR prepare a set of kernel code and kernel data for each
%user program execution 
%


%% Additionally, the Multik that customizes kernel image to create the
%% profile contains the necessary kernel code for each traced application

%% %
%% KHide restricts the granularity of software diversity techniques for
%% kernel code and kernel data with hardware virtualization
%%   \cite{gionta16cns}.
 %

%The study conducted in \cite{gionta16cns}, \reduline{the authors proposed that
 

%{\bf Flow Integrity:}
%\reduline{
%The memory protection of kernel that requires the flow integrity for
%the prevention of memory corruption.}
%The flow integrity for the prevention of memory corruption
%
%The control flow integrity (CFI) taht prevents the malicious behavior
%behind the benign program flow \cite{abadi05ccs}.
%
%
%Additionally, in \cite{?}, \reduline{
%  the authors proposed the data flow integrity that
%  verifies the benign relationship between data definition and usage for the running program.
%}

%
%\reduline{To mitigate memory corruption, the randomization and
%  granular privilege handling for the kernel resilience to enforce the
%  runtime kernel protection.}
%Kernel memory protection is 
%approach that realizes mitigation of memory corruption.
%The memory corruption

%memory modification.
%proposes the control flow
%enforcement, the randomization, and granular privilege handling for
%the runtime kernel protection.
%the kernel resilience to enforce the

%KCoFI corresponds with CFI for the kernel processing that requires the
%asynchronous behavior to handle the interruption and context switch
%\cite{criswell14sp}.
%
%In \cite{davi16ndss}, \reduline{ the authors demonstrated the
%% The randomizing protection of page table structure camouflages the
%% position of the kernel page table from malicious activity
%% \cite{davi16ndss},
%  the randomization of 
%position that protects the entire 
%
%
%% XPFO manages the attribution separation of the page between the
%% user and the kernel modes to protect direct mapping region attacks
%% \cite{kemerlis14usenix}.
  %protects the
%kernel to manage the page attribution distinction between the user and
%the kernel modes through direct mapping region attacks.
%
%% kR\^{}X that controls the exclusive mechanism across
%% with the access and the execution privilege of the kernel code and
%% kernel data \cite{pomonis17eurosys}.

%% Additionally, xMP provides switching of the visible virtual memory
%% region between the user and the kernel modes for the guest OS with the
%% hypervisor \cite{proskurin20sp}.

%provides the exclusive mechanism between access and execution of
%the kernel code and kernel data.
%

%%   the minimum mapping of  kernel memory for the generation of 
%%   profiles the necessary kernel code generated for
%%   the customized kernel image for each application
%%   authors designed 
%% Multik that reduces the available kernel code to
%% create the minimum mapping of kernel memory for each
%% application.

%Moreover, Multik profiles the necessary kernel code generated for a
%customized kernel image for each application \cite{kuo19arxiv}.


%% \reduline{[N-0X]
%%   Kernel security researches have yielded multiple
%%   %software and hardware memory security mechanisms
%%   memory isolation and memory protection mechanisms against
%%   potential threats and practical attack techniques
%% }
%% %
%% Figure \ref{fig:kernel_protection_taxonomy}
%% %
%% \reduline{indicates an overview of the kernel memory security research
%%   taxonomy, summarizing previous security mechanisms.  }

%% \begin{figure}[hb]
%%   \begin{center}
%%     \hspace{-6.0ex}
%%     \includegraphics[bb=0 0 1749 914, scale=.151]{./imgs/A01X_06_screen_shot_2021-08-26_16.39.34.png}
%%     %  \vspace{-4.5ex}
%%   \end{center}    
%%   \caption{
%%     %
%%     Overview of the taxonomy of kernel memory security mechanism.
%%     %Overview of kernel protection taxonomy.
%%     %
%%   }

%%   \label{fig:kernel_protection_taxonomy}
%% %\vspace{-5.0ex}        

%% \subsection{Memory Isolation}
%% \reduline{
%%   The runtime memory separation techniques adopt the memory isolation
%%   using hardware or software against directory attack threats from the
%%   kernel or user modes.
%% }

%% {\bf Isolation at Hardware:}
%% %
%% \reduline{
%%   The hardware features support a memory isolation mechanism.
%% }
%% %The CPU features support a memory isolation mechanism. A trusted
%% %
%% The study conducted in \cite{lee19arxiv,marcela19arxiv},
%% \reduline{ 
%% a trusted execution environment (TEE) executes the kernel in the secure
%% memory region to mitigate kernel attacks from a non-secure memory
%% region.}
%% %
%% Additionally, in \cite{ge14most} the authors proposed the Sprobes taht
%% adopts CPU security features and a TEE for the determining of kernel
%% integrity.
%% %
%% In \cite{gravani19arxiv},
%% \reduline{ 
%% the authors proposed the IskiOS adopts a CPU
%% memory protection feture that restricts a specific memory region
%% related to the kernel virtual memory.}
%% %

%% The approach with virtualization, in \cite{hua18atc}, \reduline{ the
%%   authors adopt hardware virtualization is available for the
%%   separation of the kernel virtual memory to achive low overhead.}
%% %
%% The study conducted in \cite{gionta16cns}, \reduline{the authors proposed that
%% KHide restricts the granularity of software diversity techniques for
%% kernel code and kernel data with hardware virtualization.}
%% %
%% Moreover, the authors in \cite{proskurin20sp} \reduline{ designed taht xMP provides
%% switching of the visible virtual memory region between the user and
%% the kernel modes for the guest OS with the hypervisor.}

%\subsection{Memory Isolation}
%% {\bf Isolation at Software.}
%% %
%% %\reduline{The memory isolation in the kernel security and dependancy.}
%% %
%% %
%% %
%% %\reduline{
%% To mitigate meltdown side-channel attacks, KPTI that provides
%% dedicated page tables for the user and kernel modes
%% \cite{gruss17ssos}
%% %
%% and the another approach with virtualization is available for the
%% separation of the kernel virtual memory to achive low overhead
%% \cite{hua18atc}.

%User space driver contains fault tolerance mechanism that separates
%the device drivers execution from main kernel processing
%
%In 
%\reduline{
%iKernel adopts virtual machine monitor that separates buggy devices on virtual machines 
%
%In 
%\reduline{
%SIDE makes the dedicated
%page table for each driver with interaction mechanisms between kernel
%and drivers 
%
%virtual address separation or kernel code
%behavior restrictions 

%
%% Moreover, the authors in \cite{proclocal} depevelod Proclocal that
%% allocates dedicated pages \reduline{to preserve} the kernel data for
%% each user process.
%% %
%% In \cite{sci}, the authors developed SCI that creates an isolated page
%% table to execute \reduline{kernel codes of} system calls during kernel
%% processing.
